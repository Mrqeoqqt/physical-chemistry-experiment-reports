% !Mode:: "TeX:UTF-8"
% !TEX program  = xelatex
\documentclass[a4paper]{article}
\usepackage{amsmath}
\usepackage{amssymb}
\usepackage{ctex}
\usepackage{graphicx}
%\usepackage{braket}
\usepackage[european]{circuitikz}
\usepackage{multirow}
\usepackage{geometry}
\usepackage{float}
\geometry{left=2.5cm,right=2.5cm,bottom=2.5cm,top=2.5cm}
\title{物理化学实验: 电动势的测定及其应用}
\author{薛明怡\quad 151250177\quad 化学化工学院}
\date{\today}
\begin{document}
\maketitle
%%\tableofcontents
%%\bibliographystyle{unsrt}
\section{实验目的}
\begin{enumerate}
\item 掌握银电极, 银--氯化银电极的制备方法.
\item 掌握电位差计的使用方法.
\item 掌握对消法测定电池电动势.
\item 掌握可逆电池电动势的应用.
\item 对实验结果进行讨论并设计拓展实验.
\end{enumerate}
\section{实验原理}
\subsection{原电池}
原电池是由两个``半电池''组成, 每一个半电池中包含一个电极和
相应的电解质溶液. 不同的半电池可以组成各种各样的原电池.
\par
\subsection{电池电动势}
电池电动势为组成该电池的两个半电池的电极电势的代数和. 
已知一个半电池的电极电势, 通过测定电动势, 即可求得另一半电池的电极电势.
电极电势包括界面电势差(扩散双电层模型), 接触电势和液接电势等.
\subsection{双液电池和盐桥}
如果两个半电池的电极反应所涉及的反应离子不能由同一种电解质提供, 
而需要用到两种电解质溶液时, 需要将两种电解质溶液隔开(通常使用素瓷烧杯或盐桥), 
这种电池称为双液化学电池.
\par 
盐桥是充满正, 负离子迁移数十分接近的高浓度电解质通道, 能有效减小
液接电势. $1\% - 2\%$琼脂的饱和$KCl$盐桥可以消除液接电势, 
其原因是$K^{+}$和$Cl^{-}$的迁移数十分接近, 而$KCl$的浓度远大于
其他电解质, 扩散和迁移主要由$K^{+}$和$Cl^{-}$完成, 因此液接电势的
数值可以降低到$1 - 2mV$, 在一般的电动势测量中, 这一微小数值可忽略不计.
\subsection{对消法}
如果选用伏特计测定电池电动势, 电路中存在的电流和内电阻会产生内电势降, 
测得的只能是两电极的端电压, 数值小于电池电动势. 所以用伏特计无法测出电动势. 
对消法测定电动势使用的仪器为电位差计, 在电流接近0的条件下测定(可逆条件), 
如下图:
\begin{figure}[H]
	\centering
	\includegraphics[width = 0.30\paperwidth]{fig/principle.png}
	\caption{对消法测定电池电动势原理}\label{wf}
\end{figure}
\begin{equation}
	\centering
	\begin{aligned}
		E &= (R_{0} + R_{i})I\\
		U &= R_{0}I\\
		\frac{U}{E} &= \frac{R_{0}}{R_{0}+R_{i}}\\
		R_{0}&\to \infty, E\approx U\\
		E_{x} &= E_{s.c}\frac{AC}{AH}\\
	\end{aligned}
\end{equation}
\subsection{电动势测定的应用}
\begin{itemize}
	\item 求难溶盐的活度积.
	\item pH的测定.
	\item 求电解质溶液的平均活度因子.
	\item 离子选择性电极和化学传感器.
	\item 细胞膜与膜电势.
\end{itemize}
在本实验中需要求难溶盐的活度积和测定未知溶液的pH.
\begin{equation}
	\centering
	\begin{aligned}
		E &= E^{\theta} - \frac{RT}{F}ln\frac{1}{a_{Ag^{+}}\cdot a_{Cl^{-}}}\\
		E^{\theta} &= \frac{RT}{F}ln\frac{1}{K_{sp}}\\
		lgK_{sp} &= lg(a_{Ag^{+}}\cdot a_{Cl^{-}}) - \frac{EF}{2.303RT}\\
		\phi_{} &= \phi_{C_{12}H_{10}O_{4}}^{\theta} - \frac{RT}{2F}ln(\frac{a_{C_{6}H_{6}O_{2}}}{a_{C_{6}H_{4}O_{2}}a_{H^{2}}^{2}})\\
			& = \phi_{C_{12}H_{10}O_{4}}^{\theta} - \frac{2.303RT}{F}pH\\
		pH &= \frac{\phi_{C_{12}H_{10}O_{4}}^{\theta}-E-\phi_{Hg,Hg_{2}Cl_{2}}}{2.303RT/F}
	\end{aligned}
\end{equation}
\section{仪器与药品}
\begin{enumerate}
    \item \textbf{仪器:} 数字电位差综合测试仪, 韦斯顿标准电池, 稳流电源, 铁架台, 去离子水洗瓶, 废液烧杯, 
    $50mL$烧杯2只, $10mL$烧杯4只, 盐桥支架, 电阻炉, U型管, 磁力搅拌加热器.
    \item \textbf{药品:} 盐桥液, $Ag$电极$3$支, $Pt$电极$2$支, 饱和甘汞电极, 电池盒, 
	饱和$KCl$溶液, 醌氢醌固体, 未知pH溶液, $0.01mol/L~KCl$溶液, $0.1mol/L~HCl$溶液, 
	$1mol/L~HCl$溶液, $0.01mol/L~AgNO_{3}$溶液, $0.1mol/L~AgNO_{3}$溶液, 镀银液.
\end{enumerate}
\section{实验步骤}
\subsection{制备电极}
\subsubsection{制备银电极}
\begin{itemize}
	\item 用细砂纸将三支银电极打磨光亮, 用去离子水冲洗, 擦干备用.
	\item 将一支银电极和一支铂电极固定到铁架台上, 在$10mL$小烧杯中加入
	约$8mL$镀银液, 将两支电极浸入镀银液中.
	\item 银电极接入稳流电源负极, 铂电极接入正极.
	\item 打开稳流电源开关, 调节电流至$3mA$, 镀银$30min$.
	\item 同法电镀另两支银电极, 放入去离子水中保存.
	\item 为了减少表面电势差的影响, 使用前银电极银棒处接触几秒钟.
\end{itemize}
\subsubsection{制备银-氯化银电极}
\begin{itemize}
	\item 电解液为$1mol/L~HCl$溶液, 新制银电极接入正极, 铂电极接入负极
	(与制备银电极相反).
	\item 打开稳流开关, 调节电流约为$3mA$, 通电$10min$.
\end{itemize}
\subsection{制备盐桥}
\begin{itemize}
	\item 打开电阻炉和磁力搅拌加热器开关, 加热U型管清洗液, 融化盐桥液.
	\item 用镊子从热水中取出U型管, 倒掉管内液体, 立即用滴管从一端加入
	融化的盐桥液, 一次性加满, 中间不能有气泡, 否则需要重新灌制.
	\item 制备好的盐桥斜靠在盐桥支架上自然冷却, 同法再灌制$3-4$根盐桥.
	\item 待盐桥冷却后, 向U型管两端补加少量盐桥液填平.
\end{itemize}
\subsection{校正电位差计}
\begin{itemize}
	\item 打开数值电位差综合测试仪开关.
	\item 若用内标法校正, 调节电位指示为1V. 将``测量选择''旋钮转至``内标''档,
	按``归零''键将``检零指示''调至零, 将``测量选择''转至``断''位置.
	\item 若用外标法校正, 需要使用标准电池. 将综合测试仪``外标+'' ``外标-''接线口
	分别连上标准电池正负极, 根据标准电池电势随温度变化关系式计算出当前温度下电动势数值.
	调节``电位指示''为对应电动势数值, 将``测量选择''旋钮转至``外标'', 按``归零''键, 
	将``测量选择''转至``断''位置.
\end{itemize}
\subsection{测量电池电动势}
\begin{itemize}
	\item 银电极和甘汞电极构成电池的电动势数值.
	检查甘汞电极内是否存在气泡, 特别是转角处, 若有则轻敲管壁赶走气泡; 电极内
	液面应到支管液面加液口处, 若液体不足, 可用滴管从加液口加入饱和$KCl$溶液.\\
	将盛有约$5mL$饱和$KCl$溶液的$10mL$烧杯放入电池盒中, 插入饱和甘汞电极, 接入测试仪``测量-''插口; 
	将盛有约$5mL~0.01mol/L~AgNO_{3}$溶液的$10mL$烧杯放入电池盒中, 插入银电极, 接入测试仪``测量+''插口. \\
	将电位指示调至$0.5V$左右, 测量选择调至``测量''再转回``断'', 
	根据``测量''状态时``检零指示''显示的零点差值调节``电位指示''数值, 实际值约等于当前``电位指示''值减去``检零指示''值.
	然后调至``测量''再转回``断'', 再进一步调节``电位指示'', 直至校零指示为0. 此时``电位指示''为该电池电动势.
	\item 更换电极溶液, 将一支新制银电极插入$0.01mol/L~AgNO_{3}$溶液的$10mL$烧杯中, 
	在$0.01mol/L~KCl$溶液的$10mL$烧杯中滴加$2$滴$0.1mol/L~AgNO_{3}$溶液, 边滴边搅拌, 插入
	另一支新制银电极, 在两烧杯溶液中插入一根新盐桥. $AgNO_{3}$溶液中的银电极接入``测量+'', 
	$KCl$溶液中的银电极接入``测量-''. 按上述步骤操作测量电极电动势. 
	\item 醌氢醌电极制备时往未知pH溶液中加入少量醌氢醌固体, 搅拌后放置几分钟达到溶解平衡, 
	插入铂电极. 在两烧杯溶液中插入一根新盐桥, 铂电极接``测量+'', 甘汞电极接``测量-''. 测量此电极电动势.
	\item 将银电极插入$0.1mol/L~AgNO_{3}$溶液接``测量+'', 银--氯化银电极插入$0.1mol/L~HCl$溶液接``测量-'', 
	插入盐桥. 测量此电极电动势.
	\item 测试完毕, 盐桥放入电阻炉上盛去离子水的烧杯中, 加热煮沸清洗, 中间换几次水. 
	清洗其他仪器并复原.
\end{itemize}
\section{注意事项}
\begin{enumerate}
	\item 盐桥的制备过程.
	\item 保证测定过程中电流基本为0.
	\item 测量速度要快.
	\item 正负极不要接反.
\end{enumerate}
\section{拓展实验}

\newpage
\section{数据处理}

\newpage
\section{思考与讨论}
\begin{enumerate}
	\item 将电池放置一段时间看看电动势发生什么变化?
	\item 测量电池的电极电动势需要哪些条件? 误差有哪些?
\end{enumerate}
\newpage
\section{原始数据记录}
\begin{table}[H]
	\caption{电动势测定数据记录}
	\begin{center}
		\begin{tabular}{l|l|l}
			\hline
			电池\quad\quad& 电动势/V \quad\quad\quad\quad& 测定温度 $^\circ$C\quad\quad\\
			\hline
			1 &	&	\\
			&	&	\\
			\hline
			2 &	&	\\
			&	&	\\
			\hline
			3 &	&	\\
			&	&	\\
			\hline
			4 &	&	\\
			&	&	\\
			\hline
		 \end{tabular}
	\end{center}
\end{table}
%%\bibliography{ref}
\end{document}