% !Mode:: "TeX:UTF-8"
% !TEX program  = xelatex
\documentclass[a4paper]{article}
\usepackage{amsmath}
\usepackage{amssymb}
\usepackage{ctex}
\usepackage{graphicx}
%\usepackage{braket}
\usepackage[european]{circuitikz}
\usepackage{multirow}
\usepackage{geometry}
\usepackage{float}
\geometry{left=2.5cm,right=2.5cm,bottom=2.5cm,top=2.5cm}
\title{物理化学实验15: 流动法测定$\gamma-Al_2O_3$小球催化乙醇脱水的催化性能}
\author{薛明怡\quad 151250177\quad 化学化工学院}
\date{\today}
\begin{document}
\maketitle
%%\tableofcontents
%%\bibliographystyle{unsrt}
\section{实验目的}
\begin{enumerate}
\item YYY
\end{enumerate}
\section{实验原理}
\subsection{ZZZ}
\section{仪器与药品}
\begin{enumerate}
    \item \textbf{仪器:} XXX
    \item \textbf{药品:} XXX
\end{enumerate}
\section{实验步骤}
\subsection{实验准备}
XXX
\subsection{YYY}
\subsection{收尾}
清洗整理仪器.
\section{拓展实验}

\newpage
\section{数据处理}

\newpage
\section{思考题}
\begin{enumerate}
	\item 自催化发生在哪一个过程? 自催化剂是何种物质?
	\item 如何理解本实验为开放系统? 并且远离平衡态?
	\item 两个稳态是哪两个状态?
	\item 实验现象的观察和解释.
	\item 电极上发生的反应?
	\item 同心圆的形成机制?
\end{enumerate}
\newpage
\section{原始数据记录}
\begin{table}[H]
	\begin{center}
		\begin{tabular}{l|l|l|l|l|l}
			\hline
			温度/$^\circ$C&  诱导期/s&  $\phi_1$/mV|$T_{1}$/s & $\phi_2$/mV|$T_{2}$/s & $\phi_3$/mV|$T_{3}$/s & $\bar{\phi}$/mV|$\bar{T}$/s\\
			\hline
			30&  &  &  &  &  \\
			\hline
			35&  &  &  &  &  \\
			\hline
			40&  &  &  &  &  \\
			\hline
			45&  &  &  &  &  \\
			\hline
			50&  &  &  &  &  \\
			\hline
		 \end{tabular}
	\end{center}
\end{table}
%%\bibliography{ref}
\end{document}